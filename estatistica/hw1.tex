\documentclass[11pt]{article}
\usepackage[utf8]{inputenc}  
\usepackage[T1]{fontenc}
\usepackage[francais]{babel}
\usepackage{amsmath,textcomp,amssymb,geometry,graphicx,enumerate}
\usepackage{algorithm} % Boxes/formatting around algorithms
\usepackage[noend]{algpseudocode} % Algorithms
\usepackage{hyperref}
\usepackage{stmaryrd}
\usepackage{mathtools}
\usepackage{cleveref}
\hypersetup{
    colorlinks=true,
    linkcolor=blue,
    filecolor=magenta,      
    urlcolor=blue,
}

\def\Name{João Carabetta}  % Your name
\def\SID{Lycée Lakanal}
\def\Login{Estatítica} % Your login (your class account, cs170-xy)
\def\Homework{1} % Number of Homework
\def\Session{}


\title{Homework 1}
\author{\Name, \texttt{\Login}}
\markboth{\Session\   \Name}{\Session\  \Name, \texttt{\Login}}
\pagestyle{myheadings}
\date{}

\newenvironment{qparts}{\begin{enumerate}[{(}a{)}]}{\end{enumerate}}
\def\endproofmark{$\Box$}
\newenvironment{proof}{\par{\bf Proof}:}{\endproofmark\smallskip}

\textheight=9in
\textwidth=6.5in
\topmargin=-.75in
\oddsidemargin=0.25in
\evensidemargin=0.25in


\begin{document}
\maketitle


\section*{Exercício 1}
\begin{qparts}

\item Assuma que $X_1$ $\sim$ Poisson($\mu$) e encontre a transformação estabilizadora de variância para a média.$\newline$


Dado que a $X_1$ $\sim$ Poisson($\mu$) então $Var(\mu) = \sigma^2(\mu) = \mu$ logo $\sigma(\mu) = \pm \sqrt[]{\mu}.\newline$ 

Portanto, $|g'(\mu)|\sigma(\mu) = c \Rightarrow |g'(\mu)|\sqrt[]{\mu} = c \Rightarrow g'(\mu) = \pm \frac{c}{\sqrt[]{\mu}}\newline$ 

Assim, $g(\mu) = \pm 2c\sqrt[]{\mu} + d, d \in R$.
$\newline$
\item Mostre que $\sqrt[]{n}(g(Y_n)-g(\mu)) \rightarrow N(0, 1/4).\newline$

Sabendo que $\sqrt[]{n}(g(Y_n)-g(\mu)) \rightarrow N(0, \sigma(\mu)^2g'(\mu)^2)$, então, devemos escolher $c$ tal que, $c = \sigma(\mu)g'(\mu) = 1/2$
$\newline$

\item Qual a importância prática desta transformação? (explique sucintamente ou apresente um exemplo)

\end{qparts}

\section*{Exercício 2}


\begin{qparts}
\item Mostre que $\sqrt{n}(g(Y_n)-g(\mu))=o_p(1)$.


Supondo que existe $g(y)$ então, podemos expandir $g$ em Taylor torno de $Y_n$:
\begin{equation}
\label{equation:TaylorSecond}
	g(Y_n) = g(\mu) + g'(\mu)(Y_n - \mu) + \frac{g''(\mu)}{2}(Y_n - \mu)^2 + R,
\end{equation}
onde $R$ é o resíduo da expansão. Assim, se operando os termos e multiplicando por $\sqrt{n}$ temos:
\begin{equation}
	\sqrt{n}(g(Y_n) - g(\mu)) = \frac{g''(\mu)}{2}\frac{\sqrt{n}}{n^2}\Big( \sum^{\infty}_{i=0}X_i} - n\mu\Big)^2 + \sqrt{n}R.
\end{equation}
Se observarmos os termos, veremos que $g''(\mu)=O(1)$, $\frac{\sqrt{n}}{n^2}=o(1)$, $\Big( \sum^{\infty}_{i=0}X_i} - n\mu\Big)^2=o_P(1)$ e $\sqrt{n}R=o_p(1)$. Portanto, o produto deles é $o_p=1$. Finalmente,
\begin{equation}
	\sqrt{n}(g(Y_n) - g(\mu)) = o_p(1)
\end{equation}

\item Mostre que $n(g(Y_n)-g(\mu)) \xrightarrow[]{d} \frac{1}{2}g''(\mu)\sigma^2V$, onde $V \sim \chi_1^2$.


Continuando da Eq.\ref{equation:TaylorSecond}, mas dessa vez multiplicando por $n$ os dois lados e por $\frac{\sigma^2}{\sigma^2}$ o lado direito temos,
\begin{equation}
	n(g(Y_n) - g(\mu)) = \sigma^2\frac{g''(\mu)}{2} \frac{n\big( Y_n-\mu\big)^2}{\sigma^2} + nR.
\end{equation}

Sabendo que $\frac{n\big(Y_n-\mu\big)^2}{\sigma^2} \xrightarrow{d} \chi^2_1$ pelo Teorema Central do Limite e $nR=o_p(1)$, então, pelo Teorema de Slutsky temos,
\begin{equation}
	n(g(Y_n) - g(\mu)) \xrightarrow[]{d} \frac{1}{2}g''(\mu)\sigma^2\chi_1^2
\end{equation}

\item Suponha que $X_1 \sim Bernoulli(p)$. Ache a distribuição de $g(Y_n) = Y_n(1-Y_n)$.

Temos que $EX_1=p$ e $Var(X_1)=p(1-p)$ para uma distribuição Bernoulli. Assim, podemos usar $g(Y_n) = Y_n(1-Y_n)$ para estimar a $Var(X_1)=p(1-p)$. Assim, 

\begin{subequations}
	\begin{align}
	g'(Y_n) &= 1-2Y_n \\ 
	g''(Y_n) &= -2
	\end{align}
\end{subequations}

mas, para como $Y_n$ estima $p$, $g'(p) = 0$ quando $p = 1/2$. Portanto,
para $p\neq1/2$ usamos Método Delta de Primeira Ordem,

\begin{subequations}
	\begin{align}
		\sqrt{n}(g(Y_n) - g(p)) &\xrightarrow[]{d} N(0, \sigma^2 g'(p)^2)   \Rightarrow \\
		g(Y_n) &\xrightarrow[]{d} N(g(p), \frac{\sigma^2 g'(p)^2}{n}) \label{equation:binomialdelta1}
		\Rightarrow \\
		g(Y_n) &\xrightarrow[]{p} p(1-p)
	\end{align}
\end{subequations}



 para $p=1/2$ é melhor usar o Método Delta de Segunda Ordem para aproximar a distribuição. Assim, 

\begin{subequations}
\begin{align}
	n(g(Y_n) - g(p)) &= \frac{1}{2}g''(\mu)\sigma^2\frac{n\big( Y_n-\mu\big)^2}{\sigma^2} \Rightarrow \\
	ng(Y_n) - n[p(1-p)] &= \frac{1}{2} (-2) p(1-p) \frac{n\big( Y_n-\mu\big)^2}{\sigma^2} \Rightarrow \\
	g(Y_n) &= p(1-p)\Big(1 -\frac{1}{n}\frac{n\big( Y_n-\mu\big)^2}{\sigma^2}\Big)
	\xRightarrow{Slutsky}\\
	g(Y_n) &\xrightarrow[]{d} p(1-p)\Big(1 -\frac{1}{n}\chi_1^2\Big) \label{equation:binomialdelta2}
	\xRightarrow{}\\
	g(Y_n) &\xrightarrow[]{p} p(1-p)
	\end{align}
\end{subequations}


\end{qparts}

%%%%%%%%%%%% EXECICIO 3
\section*{Exercício 3}
\begin{qparts}
	
%%%%%%%%%%%% ITEM A
\item $\hat{p}=\sum_{i=1}^24x_i=11/24$.


Usando o método \textbf{Delta}:

Para $p\neq1/2$, podemos usar a Eq.\ref{equation:binomialdelta1}, com $\sigma^2=p(1-p)$ e $g'(p)=(1-2p)$

\begin{subequations}
	\begin{align}
		g(P_n) &\xrightarrow[]{d} N(g(p), \frac{\sigma^2 g'(p)^2}{n}) \Rightarrow \\
		g(P_n) &\xrightarrow[]{d} N\left(p(1-p), \frac{p(1-p) (1-2p)^2}{n}\right) \Rightarrow \\
		g(P_n) &\xrightarrow[]{d} N\left(0.24826, 0.00004\right)
	\end{align}
\end{subequations}

Usando o método \textbf{Bootstrap}:

Para o método Bootstrap, confira 

%%%%%%%%%%%%% ITEM B
\item $\hat{p}=\sum_{i=1}^2x_i=12/24$.

Usando o método \textbf{Delta}:

Agora, para $p=1/2$ temos que usar Eq.\ref{equation:binomialdelta2}. Assim, a esperança é,

\begin{subequations}
	\begin{align}
		Eg(Y_n) &= E\left(p(1-p)\left(1-\frac{\chi^2_1}{n}\right)\right) \Rightarrow \\
		Eg(Y_n) &= p(1-p)(1-1/n) \Rightarrow \\
		Eg(Y_n) &= 1.043478260
	\end{align}
\end{subequations}

e a variância é,

\begin{subequations}
	\begin{align}
		Var(g(Y_n)) &= Var\left(p(1-p)\left(1-\frac{\chi^2_1}{n}\right)\right) \Rightarrow \\
		Var(g(Y_n)) &= p^2(1-p)^2Var(1-\frac{\chi^2_1}{n}) \Rightarrow \\
		Var(g(Y_n)) &= \frac{2p^2(1-p)^2}{n^2} \Rightarrow \\
		Var(g(Y_n)) &= 0.001
	\end{align}
\end{subequations}

Logo, $g(Y_n) = 1.043 \pm 0.001$.


%%%%%%%%%%%%% ITEM C
\item Suponha que o valor verdadeiro de $p=0.24$. Calcule a variância "exata" usando simulação e discuta os resultados acima. Qual método você se sentiria mais confiante em empregar?

	
\end{qparts}


\end{document}

